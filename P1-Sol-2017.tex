\documentclass[12pt]{article}
\usepackage{graphicx,url}
\usepackage[brazil]{babel}   
\usepackage[utf8]{inputenc}  
\usepackage{verbatim}
\usepackage{listings}
\usepackage{xcolor}
\usepackage{amssymb}
\definecolor{verde}{rgb}{0,0.5,0}
\title{PROVA 1}

\begin{document} 

\maketitle

\textbf{QUESTÃO 1}\\

\textbf{a)}
\begin{itemize}
\item Se $x\in(-\infty,-1)$ temos:\\
$-(x+1)-(x-2)=4$\\
$-2x=3$\\
$x=- \frac{3}{2}$\\
\item Se $x\in[-1,2)$ temos:\\
$+(x+1)-(x-2)=4$\\
$3=4$\\
Portanto $\not\exists x\in \mathbb{R}$ nesse intervalo que seja solução.\\
\item Se  $x\in[2,+\infty)$ temos:\\
$+(x+1)+(x-2)=4$\\
$2x=5$\\
$x=\frac{5}{2}$\\

Portanto $x=- \frac{3}{2}$ ou $x=\frac{5}{2}$.\\
\end{itemize}

\textbf{b)} 
Estudando os sinais de cada parcela temos:
\begin{itemize}
\item $x\geq0$\\
\item $2x-1\geq0 \Leftrightarrow x\geq\frac{1}{2}$\\
\item $(x-2)^3\geq0 \Leftrightarrow x-2\geq0 \Leftrightarrow x\geq2$\\
\item $-3x+\geq0 \Leftrightarrow x\leq\frac{2}{3}$\\
\item $(x-1)^2\geq0 \Rightarrow x\geq 0$ para todo $x\in\mathbb{R}$\\
\end{itemize}
Fazendo o estudo de sinais temos que a inequação acima é maior que zero quando\\ $x\in(-\infty,0]$ ou $x\in[\frac{2}{3},2]$.\\

\textbf{c)} 
Devemos ter $|2x+1|\geq|1-x|$, então:
\begin{itemize}
\item Se $x<-\frac{1}{2}$ temos:\\
$-(2x+1)\geq1-x$\\
$x\leq-2$\\
\item Se $x>-\frac{1}{2}$ temos:\\
$+(2x+1)\geq1-x$\\
$x\geq0$\\
\item Devemos ter também $|2x+1|\neq0 \Rightarrow x\neq-\frac{1}{2}$ e $1-x\neq0 \Rightarrow x\neq 1$.\\
\end{itemize}
Portanto $x\in[0,1)$ ou $x\in(1,+\infty)$\\
\\
\\

\textbf{QUESTÃO 2}\\
Suponhamos que $\frac{\sqrt{2}}{\sqrt{5}}+1$ seja racional.\\
Sabemos que $1\in\mathbb{Q}$ e que a soma de racionais é racional, logo devemos ter $\frac{\sqrt{2}}{\sqrt{5}}\in\mathbb{Q}$.\\
Se $\frac{\sqrt{2}}{\sqrt{5}}\in\mathbb{Q}$ então $\exists p,q\in\mathbb{Z}$ onde $p$ e $q$ não tem fatores em comum e $\frac{p}{q}=\frac{\sqrt{2}}{\sqrt{5}}$.\\
Daí temos que $\frac{p^2}{q^2}=\frac{2}{5}\Rightarrow5p^2=2q^2$.\\
Vemos que $5p^2$ é par e como $2$ não divide $5$ temos que $p^2$ é par e consequentemente $p$ é par, pois todo quadrado de um número par é sempre par.\\
Seja então $p=2r$, temos que $5.(2r)^2=2q^2\Rightarrow5.2r^2=q^2$.\\
Daí vemos que $q^2$ é par e consequentemente $q$ é par, o que é um absurso pois $p$ e $q$ não tem fatores em comum.\\
Logo $\frac{\sqrt{2}}{\sqrt{5}}\not\in\mathbb{Q}$ e consequentemente $\frac{\sqrt{2}}{\sqrt{5}}+1$ não é racional.\\
\\
\\

\textbf{QUESTÃO 3}\\

\textbf{a)} 
$D_f=\mathbb{R}$
\\

\textbf{ b) } 
\begin{itemize}
\item se $x<1$ temos\\
$f(x)=-(x-1)-(x-2)$\\
$f(x)=-2x+3$\\
\item se $1\leq x<2$ temos\\
$f(x)=+(x-1)-(x-2)$\\
$f(x)=1$
\item se $x\geq2$ temos\\
$f(x)=+(x-1)+(x-2)$\\
$f(x)=2x-3$\\
\end{itemize}
Logo,\\

%TODO corrigir isso aqui 
\begin{comment}
$
f(x) = \left\{
\begin{array}{rcl}
-2x+31,& \mbox{se} & x<1\\
1, & \mbox{se} & 1\leqx<2\\
2x-3, & \mbox{se} & x\geq 2
\end{array}
\right.
$\\
\end{comment}


\textbf{ QUESTÃO 4}\\

\textbf{a)} Falso. Seja $x=-2$ e $y=-1$, temos que $x<y$. Mas $x^2=4$ e $y^2=1$, logo $y^2<x^2$.\\
\\
\textbf{b)} Verdadeiro.\\
\\
\textbf{c)} Falso. Seja $x=\frac{1}{4}$. Temos que $\sqrt{\frac{1}{4}}=\frac{1}{2}$ e $\frac{1}{4}\leq\sqrt{\frac{1}{4}}$.
\\
\\

\textbf{QUESTÃO 5}\\

\textbf{a)} Se $x$ é a idade da pessoa temos que $x+(x-25)=95\Rightarrow x=60$. Portanto a pessoa tem direito de se aposentar aos $60$ anos.\\
\\
\textbf{ b) } Seja $x$ a idade que a pessoa começa a trabalhar e $f(x)$ a idade que a pessoa tem o direito de se aposentar. Sabemos que a soma da idade que a pessoa tem o direito de se aposentar com o tempo de serviço deve ser igual a 95.\\
Sabemos que o tempo se serviço é dado pela idade da pessoa menos a idade que ela começou a trabalhar, então $f(x)+(f(x)-x)=95\Rightarrow f(x)=\frac{1}{2}(95+x).$
\end{document}